
\documentclass{resume} % Use the custom resume.cls style

\usepackage[utf8]{inputenc}
\usepackage[left=0.5 in,top=0.5in,right=0.9 in,bottom=0.5in]{geometry} % Document margins
\newcommand{\tab}[1]{\hspace{.2667\textwidth}\rlap{#1}} 
\newcommand{\itab}[1]{\hspace{0em}\rlap{#1}}
\name{\LARGE{Tristán Ignacio Gramsch Calvo}} % Your name
\address{Phone (312) 613-0342}
\address{email: tristangramsch@gmail.com}
\address{62 Elm Street, Apt 2, Somerville MA 02144}

\begin{document}

\begin{rSection}{Profile}

I am a data scientist interested in health economics. My area of expertise is drug addiction. I am pursuing job offers in data-driven organizations that do social good. For years, I have studied the dire overdose epidemic in the US. I want to join people who want to put an end to it. 

\end{rSection}

\begin{rSection}{Methodological Training}

Advanced technical skills: R, SQL, Git, Markdown\\
Intermediate technical skills:  LaTex, MAXQDA, Geoda – Spatial Data\\
Relevant Skills: Data architecture; Proficiency in managing productive remote teams; Proficiency in research, data collection and analysis; Human subject experiments; Statistical analysis

\end{rSection}

\begin{rSection}{Education}

{\bf University of Chicago, Master of Arts Program in the Social Sciences} \hfill {2018 - 2019}
\\
Thesis title: “Distant, disposable, or similar: why drug users discuss hurtful topics with people they do not consider close".\\
Advisor: Reuben Miller, PhD

{\textbf{Universidad de Chile, BS in Sociology}}  \hfill 2012 - 2018\\
Thesis title: “The importance of open communication in Substance Dependent Individuals' rehabilitation a case study”.\\
Advisor: Rodrigo Asún, PhD\\
Academic excellence award (top 10\% cohort performance)
 
 \end{rSection}

\begin{rSection}{Work and Research Experience}
{\textbf{Cook County Assessor's Office}}  \hfill 2020 - Present\\
Fellow data scientist\\
Led an all-encompassing reporting infrastructure epic. The reporting infrastructure is capable of producing 126 dynamic 'ratio studies' per year. Every combination of parameters gets a single report. Developed Shiny application for user to activate the infrastructure.\\
Responsible for creating statistical models to predict rent prices of apartments in Cook County\\
Deployed an automatized integrity check of department's data

{\textbf{University of Chicago}}  \hfill 2018 - 2019\\
Graduate Student Researcher, Social Sciences Division\\
Managed full cycle independent research project; Gained knowledge of IRB protocols\\
Conducted in depth interviews to assess with whom and why drug addicts vent their emotions; Analyzed data using MAXQDA\\
\textbf{Graduate Coursework \& Projects}\\
Conducted causal inference analysis using propensity score based methods. Observed the true effect of a small class on learning outcomes\\
Used GeoDA to visualize a the distribution of drug related crime in Chicago.

{\textbf{Research Assistant}}  \hfill 2017 - 2018\\
For Rodrigo Asún, Sociology Department, 'Universidad de Chile'\\
Conducted literature research to formulate explanations of student’s social movement\\
Analyzed newspaper text using Excel and proposed methodological corrections to the approach

{\textbf{Universidad de Chile}}  \hfill 2017 - 2018\\
Undergraduate Student Researcher, BS Thesis Project, Sociology program\\
Conducted semi-structured interviews to assess talking behaviors of drug addicts\\
Criticized American Psychological Association’s manual (APA) for not considering social network factors to define drug addiction

\end{rSection}

\begin{rSection}{Academic Awards}
{‘Becas Chile’ Award (prestigious scholarship covering full tuition/expenses), University of Chicago} \hfill 2018\\
{Merit-Based scholarship of Half Tuition, University of Chicago} \hfill 2018\\
{Outstanding thesis award for BS thesis, Universidad de Chile} \hfill 2017\\
{International Innovation Corps Fellowship, Harris School of Public Policy at The University of Chicago. Expected to do research to evaluate and create solutions for social policy} \hfill {February, 2020 - present}

\end{rSection}

\begin{rSection}{Additional training and academic activities}

{{Completed 'How to manage a remote team', lectured by Gitlab}}  \hfill 2020\\
{{MAPSS/MACS Summer Math Camp, University of Chicago}}  \hfill 2018\\
{{Board Member, ‘Némesis’ Student Academic Yearly Magazine - Social Sciences Faculty, Universidad de Chile}}  \hfill 2013-2018\\
{{Co-Founder, System Theory and Complexity Study Group, Universidad de Chile}}  \hfill 2016-present

\end{rSection}

\begin{rSection}{Publications}
Gramsch, T (2020). Psychiatric blindspot: how widespread definitions of drug addiction miss social network factors. ICSHI 2021: 15 Conference procedures. To be published in 2021. 
Baecker, D. (2017). Teorías sistémicas de la comunicación. Revista Mad, (37), 1-20. English translation by Tristán Gramsch\\
Gramsch, T (2016). Análisis crítico de Estándares de Evaluación para América Latina y el Caribe, propuesta de la Red de Seguimiento, Evaluación y Sistematización de Latinoamérica y el Caribe, 2016. Red Internacional de Evaluación, 1-25\\
Gramsch, T (2015). El libro de la Comunicación. Editorial Isidora Cartonera. Santiago, Chile\\
Gramsch, T (2013). Desastres Socionaturales: La sequía en la provincia de San Felipe y cómo afecta a la población de Las Coimas. Cuadernos de Crisis y emergencias, (13) Vol. 1, 95-102

\end{rSection}

\begin{rSection}{Conference presentations}

\textbf{“Psychiatric blindspot: how widespread definitions of drug addiction miss social network factors”}\\
To be presented and published at the 'ICSHI 2021: 15. International Conference on Sociology of Health and Illness'

\textbf{“Building a more transparent government using Shiny”}\\
SatRday conference at Chicago, Chicago, 2020

\textbf{“Distant, disposable, or similar:  why drug users discuss hurtful topics with people they do not consider close”}\\
Netherlands Institute for the Advanced Studies in Social Sciences and Humanities Conference, 2021. Netherlands\\
2021 INGRoup Conference. USA\\
Global Community of Social Science: International Academic Conference: Contemporary Trends and Social Science Discussions, 2020. Spain -selected, not able to attend\\
2nd International Conference on Social Sciences and Arts, 2020. Thailand\\
MAPSS Academic Research Conference, 2019. University of Chicago\\
DePaul’s Oxford House research team workshop, 2019. Chicago. 

\textbf{“Un marco epistemológico interdisciplinario: uno de los desafíos de las ciencias sociales"}\\
I Research Congress of the Social Sciences Faculty, 2018. University of Chile. 

\textbf{“The importance of open communication in Substance Dependent Individuals' rehabilitation a case study”}\\
FACSO Undergrad research conference, 2017. University of Chile

\textbf{“Desastres Socionaturales: La Sequía en la Provincia de San Felipe y Cómo afecta a la población de Las Coimas”}\\
Conference 'Cuadernos de Crisis y Emergencias', 2013. Santiago

\end{rSection} 

\begin{rSection}{Teaching experience}

\textbf{Teaching assistant, Universidad de Chile}\\
“Social Theory: Society as a Social System” imparted by Marcelo Arnold\hfill 2017\\
“Qualitative Techniques II” imparted by Claudio Duarte and Pablo Cottet\hfill2016\\
“Sociological Theory I” Imparted by Rodrigo Figeroa\hfill2016\\
“Health and Society” Imparted by Marcela Ferrer\hfill2016\\
"Psychology I” Imparted by Gabriel Abarca and Danilo Sanhueza\hfill2016

\end{rSection}

\begin{rSection}{Languages}

Spanish (Native)\\
English (Full professional proficiency)\\
German (Elementary proficiency)

\end{rSection}

\end{document}

